\part{The Aircraft}
\chapter{General Description}

\chapter{Systems}

\section{Auxiliary Power Unit (APU)}
\label{sec:apu}

A Honeywell Aerospace GTCP85-180L \gls{APU} is located in the forward part of the left wheel well. The \gls{APU} is a small, self-contained, single shaft gas turbine operating at a constant speed of approximately 42.000 \gls{RPM}. A shaft driven 40 \gls{kVA} AC generator supplies power to the electrical system. During ground operation bleed air can be used for engine starting and operation of the air-conditioning systems. Using the aircraft battery to start the \gls{APU} allows for operations on remote locations, without any ground support equipment available.

%Electrical starter, control circuits and ignition are powered through the Isolated DC bus.

\subsection{Fuel System}

The APU uses the same fuel as used for the main engines. Depending on the applied load, typical fuel consumption varies between 130 \gls{lbs/hr} and 270 \gls{lbs/hr}.

\paragraph*{Fuel Supply}
Fuel is gravity feed from the No. 2 main tank through a firewall shutoff valve installed in the No. 2 dry bay. The shutoff valve prevents fuel flow anytime the \gls{APU} control switch is placed in the STOP position, the \gls{APU} fire handle is pulled or the \gls{APU} control circuits are deengerized.

\paragraph*{Fuel Control Unit}
The fuel control unit fully automatically operates the fuel pump. Using a shaft governor the amount of fuel supplied to the combustor is varied to keep the rotational speed constant. A thermostat limits the fuel flow to protect the turbine from overtemperature, effectively also limiting acceleration during starting. While extracting bleed air, temperature limiting is shifted to the \nameref{par:bleed-air-shutoff-and-load-control-valve}.

\paragraph*{Fuel Shutoff Valve}
\label{par:fuel-shutoff-valve}
At the outlet of the fuel control unit another fuel shutoff valve opens once reaching a minimum oil pressure. During startup it opens at approximately 10\% \gls{RPM}. If oil pressure is lost the valve closes and the \gls{APU} will automatically shut down.

\subsection{Oil System}

A pressurized oil system provides lubrication to gears and shaft bearings. The shaft driven oil pump delivers oil from an external reservoir to the gears and bearings. A relieve valve maintains an operating pressure within 90$\pm$10 \gls{psi} while the \gls{APU} operates at 100\% \gls{RPM}. Oil temperature is regulated by either directing oil flow through the oil cooler, or if oil temperature is below 27°C, through the oil cooler by-pass valve.

\paragraph*{Oil-pressure sequencing switch}
At approximately 15\% \gls{RPM} oil pressure is sufficient to operate the oil-pressure sequencing switch, which completes circuits to the \nameref{par:fuel-shutoff-valve} and ignition system. It prevents starting without lubrication and ensures an adequate airflow for combustion before introducing fuel and initiating ignition.

\paragraph*{Door-control oil-pressure switch}
Oil pressure below 20 \gls{psi} (equals approximately 18\% \gls{RPM}) operates the door-control oil-pressure switch allowing to close the \nameref{par:air-intake-door}. This is done to prevent collapse of the air inlet duct due to build-up of negative pressure.

%\begin{bclogo}[logo=\bclampe, ombre=true, couleurOmbre = black!80, couleurBarre=blue, marge=6]{Note}
\begin{bclogo}[logo=\bclampe, ombre=false, couleurBarre=blue, marge=18, noborder=true]{Note}
\indent
Placing the APU control switch in the "STOP" position or pulling the APU fire handle is required to complete the circuit for actually closing the \nameref{par:air-intake-door}.
\end{bclogo}

\subsection{Airflow}

\paragraph*{Air Intake Door}
\label{par:air-intake-door}

\paragraph*{Bleed Air Shutoff and Load Control Valve}
\label{par:bleed-air-shutoff-and-load-control-valve}

\subsection{Controls and Indications}

\nameref{sec:apu-panel} and \nameref{sec:bleed-air-panel}

%40 KVA AC generator, electrical power up to 20.000 feet.

%APU air inlet door closes if APU control switch is in STOP position and oilpressure decreases below approximately 20 PSI, which happens at about 18 percent \gls{RPM}.

%Starter approx. 1100 Watt. inrush current 4-8 times normal

APU bleed air valve switch on the Bleed Air Panel.

\section{Advisory, Caution and Warning System (ACAWS)}
\label{sec:acaws}

\gls{ACAWS}

\section{Electric System}

2 batteries: Utility, Avionics 24V 42Ah at C1

\#2 generator powers essential AC bus, if \#2 lost APU powers essential AC bus (APU always powers essential AC bus if online), else \#1 takes load (same side takes/assumes load). If two engines/generators on same side shutdown/lost, symmetrical engines pick up loads.

\#1 generator: Left Hand AC bus.\\
\#2 generator powers Essential AC bus.\\
\#3 generator will power the Main AC bus.\\
\#4 generator will power the Right Hand AC bus.

Gen 1-4 and APU ACAWS Indications. When the GCU detects an out of tolerance condition, and a generator switch or EXT PWR/OFF/APU switch is in ON (generator) or APU (apu), it will open the line contactor and send a system status indication to the bus interface unit BIU. The
BIU will then generate a data word to the mission computer that a failure has occurred in one of the engine generators or the APU generator. An advisory, caution, and warning system (ACAWS) text message (GEN 1,2,3, or 4 FAIL or APU GEN FAIL) will then be displayed for the first ten
seconds in reverse color (black letters on a night vision imaging system (NVIS) yellow background) followed by flashing master caution lights.

\section{Avionics System}

The \gls{BIU}s convert various signals and serve as additional backup bus controllers (and somehow the Mission Computers?).

\section{Hydraulic System}

Utility hydraulic system: wing flaps, main landing gear, normal braking, nose wheel steering.

\section{Enhanced Cargo Handling System (ECHS)}
\label{sec:echs}

\gls{MFCD}, \gls{RECP}, \gls{CAWS} 12 pairs of electric pallet locks (40 inch spacing) http://www.google.com/patents/EP0771726A2?cl=en

\subsection{Cargo Compartements}

\begin{itemize}
  \itembf{C} 245 - 281
  \itembf{D} 281 - 337
  \itembf{E} 337 - 401
  \itembf{F} 401 - 457
  \itembf{G} 457 - 517
  \itembf{H} 517 - 597
  \itembf{I} 597 - 627
  \itembf{J} 627 - 682
  \itembf{K} 682 - 737
  \itembf{L} 737 - 803 (Ramp)
  \itembf{M} 803 - 869 (Ramp)
\end{itemize}

\subsection{Locks}

\begin{enumerate}
  \item 302
  \item 342
  \item 382
  \item 422
  \item 462
  \item 502
  \item 542
  \item 582
  \item 622
  \item 662
  \item 682
  \item 803?
\end{enumerate}

% right 884px/885px
% left 80px/87px
% total 800px -> l: 112px, r: 912

\chapter{Cockpit Controls}

Blah blah

\section{Overhead Panel}

\subsection{Electrical Panel}

\begin{enumerate}
  \itembf{Battery (BTRY) switch} Connects battery bus to isolated DC bus.
\end{enumerate}

\subsection{Bleed Air Panel}
\label{sec:bleed-air-panel}

\begin{enumerate}
  \itembf{\gls{APU} Bleed Air Valve switch}{OPEN/CLSD}
  \itembf{Divider Valve switch}{CLOSE/AUTO/OPEN}
  \itembf{Wing Isolation Valve switches}{CLOSE/AUTO/OPEN}
  \itembf{Nacelle Shutoff Valve switches}{CLOSE/AUTO/OPEN}
\end{enumerate}

\subsection{Fuel Management Panel}

Transfer switches (TO/FROM)

\subsection{Engine Start and Fire Control Panel}
\label{sec:eng-panel}

\begin{enumerate}
  \itembf{Engine 1/2/3/4 Fire Handle}
  \itembf{Engine 1/2/3/4 Start switch}
    \begin{itemize}
      \itembf{MOTOR}
      \itembf{STOP}
      \itembf{RUN}
      \itembf{START}
    \end{itemize}
    Fuel pumps in respective tank operated in RUN/START and stopped in MOTOR/STOP
\end{enumerate}

\subsection{APU Panel}
\label{sec:apu-panel}

The \gls{APU}...

\begin{enumerate}
  \itembf{Control switch}
    \begin{itemize}
      \itembf{STOP} Same as 110\% overspeed switch. APU door closes if below 18\% (or oilpressure below 20 PSIG).
      \itembf{RUN} APU door opens.
      \itembf{START}{spring-loaded position} If APU door opened at least 15° starter engages
    \end{itemize}
  \itembf{\gls{EGT} indication}
  \itembf{\gls{RPM} indication}
  \itembf{Fire handle} The T-shaped fire handle
    \begin{enumerate}
      \item Control power interrupt
      \item Fuel shut-off valve closes
      \item Once the APU \gls{RPM} falls below approximately 18\% the APU air intake door is closed.
      \item Agent discharge switching available
    \end{enumerate}
\end{enumerate}

\section{Instrument Panel}

\subsection{Avionics Management Unit (AMU)}
\label{sec:amu}

See \nameref{chap:amu} for details.

\subsection{Head Down Displays (HDD)}
\label{sec:hdd}

\subsection{Autopilot}

Reference Settings
\begin{enumerate}
  \itembf{HP}
  \itembf{RAD ALT} Radar altitude
  \itembf{IAS} \gls{IAS}
  \itembf{FPA} \gls{FPA}
  \itembf{MINS}
\end{enumerate}

Mode Selections
\begin{enumerate}
  \itembf{ALT}
  \itembf{VS} Vertical Speed
  \itembf{SEL}
  \itembf{IAS} \gls{IAS}
  \itembf{HDG} Heading
  \itembf{NAV}
  \itembf{CAPS} \gls{CAPS}
  \itembf{APPR}
  \itembf{A/T} \gls{A/T}
\end{enumerate}

REF SET\\
ALT SET\\
BARO SET

\section{Flight Controls}

\subsection{Yoke}

\paragraph*{Left side}
HUD DECLUTTER\\
?PH\\
RADIO\\
STOP WATCH

\paragraph*{Front/Right side}
ELEV TRIM (Dual Rocker Switch) NOSE DN / UP\\
?\\
SYN (VERT REF SET) (Pitch Synchronization)\\
?\\
G/A (GO AROUND)

\chapter{Cargo Compartement Controls}

\section{Loadmaster Station}

\subsection{Multi Functional Control Display (MFCD)}
\label{sec:mfcd}

\paragraph*{MENU}

Shows cargo compartment graphic and CAWS messages. Use \glspl{OSB} to change page. Use this page if \gls{ECHS} is not in use.

\paragraph*{ONLOAD/CURRENT CARGO?}

\paragraph*{OFFLOAD}

\paragraph*{COMBAT OFFLOAD}

\paragraph*{AIRDROP PROGRAM}

\paragraph*{AIRDROP}

\paragraph*{JETTISON}

\paragraph*{LOCK CONTROL}

\subsection{Ramp and Emergency Control Panel (RECP)}
\label{sec:recp}

